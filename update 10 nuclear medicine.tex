\documentclass{article}
\usepackage[utf8]{inputenc}

\title{ }
\author{Submitted By Lingamgunta saikumar}
\date{26, march 2022}

\begin{document}

\maketitle

\section{Conclusion}
Nuclear medicine is a vital part of cancer treatment because it covers a physiologic gap in medical 
imaging by delivering information that is representative of function part of the diagnosis, treatment, 
and prognosis of disease.
General nuclear medicine has been shown to be effective in the treatment of breast cancer. At each 
stage of a patient's life, medicine provides diagnostic value. It is a first-line treatment for a variety of 
diseases. Examples of clinical circumstances Gamma cameras are quite versatile. outfitted with 
SPECT and/or SPECT/CT are unquestionably advantageous ideal tool for patients in a broader clinical 
setting, providing a thorough look into the current physiologic condition of bodily structures and 
processes. PET/CT has established itself as a powerful oncologic tool, and with the introduction of 
tailored radiopharmaceuticals, it is expected to see a significant increase in its use in the treatment 
of breast cancer. Future research should look at multi-acquisition (planar and SPECT) and multimodality (SPECT/CT) imaging in the full scope of physiologic breast cancer imaging to better 
understand the added utility of instrument hybridisation methods and processes in nuclear medicine 
imaging.
\end{document}
