\documentclass{article}
\usepackage[utf8]{inputenc}

\title{Role of nuclear medicine in breast cancer}
\author{Submitted By Lingamgunta saikumar}
\date{21, Jan 2022}

\begin{document}

\maketitle

\section{Abstract}
In the world, this breast cancer is the most popular and growing disease especially in present 
situations. By early detection only we can prevent this cancer. And there are so many cases that 
are handled by early detection and decrease the rate of death. And also many research works 
have been done on this breast cancer. In this paper we are discussing about detection of breast 
cancer through nuclear medicine . Routine nuclear medicine is a major contribution to a 
complete range of clinical studies such as early lesion detection and stratification; monitoring , 
guiding, progression in monitoring, recurrence or metastases and predicting response to 
therapy. Particularly recent developed instrumentation such as high-resolution device coupled 
with the diagnostic versatility of conventional cameras have reinserted nuclear medicine as a 
valuable tool in the broader clinical setting. This shows the outline of nuclear medicine, 
concluding that targeted radiopharmaceuticals and versatile instrumentation position nuclear
medicine as a powerful modality for patients with breast cancer.
\section{Introduction}
World wide, breast cancer is the most common cancer in women and there were 2.3 million 
women diagnosed with breast cancer and 685000 deaths globally in the year of 2020. As the end 
of 2020, there were 7.8 million women alive who were diagnosed with breast cancer in last 5 
years, making it the most prevalent cancer . Breast cancers occur in every country of the women 
at any age after puberty but with increasing rates in later life.
\end{document}
