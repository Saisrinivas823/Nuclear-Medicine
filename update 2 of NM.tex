\documentclass{article}
\usepackage[utf8]{inputenc}

\title{Role of nuclear medicine in breast cancer}
\author{Submitted By Lingamgunta saikumar}
\date{31, Jan 2022}

\begin{document}

\maketitle

\section{Introduction}
While changing our lifestyle factors such as excessive alcohol consumption, obesity and 
physically inactive have also contributed to increased the rate of death from breast 
cancer. Mortality rate which is caused by breast cancer is currently the second leading 
cause of death for women in developed regions. Since 2008, the support of clinical 
approaches that decreases the risk and increase early detection and treatment through 
worldwide, while frequency increases in 20% and rise in mortality by 14%. When 
incorporating nuclear medicine technology in the diagnostic plan, it can readily achievable 
by multimodality approach by using tailored management. In instruments the 
developments has been taking place such as high resolution dedicated breast device and it 
has coupled with the diagnostic versatility of conventional cameras have reinserted 
nuclear medicine as a valuable tool in clinical setting in broader range. This shows general 
nuclear medicine is a critical role in the care of women with breast cancer including 
detection and stratification, guiding treatment by using sentinel node imaging (which is 
used to determine whether cancer has spread beyond a primary tumor into your 
lymphatic system), monitoring cardiotoxicity from therapeutic regimens and 
evaluating local and global progression. In addition with that, it describes the role of 
positron emission tomography or computed tomography is discussed.
\end{document}
