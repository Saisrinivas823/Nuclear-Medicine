\documentclass{article}
\usepackage[utf8]{inputenc}

\title{Role of nuclear medicine in breast cancer}
\author{Submitted By Lingamgunta saikumar}
\date{11, Feb 2022}

\begin{document}

\maketitle

\section{Scintimammography}
Scintimammography in nuclear medicine, which utilizes a wide range of instrumentation 
applications . Especially in recent years, conventional planar scintimammography has been 
enhanced by single-photon has been enhanced by single-photon emission computed 
tomography (SPECT) and hybrid SPECT/CT . while in hybrid SPECT/CT adds clinical value by 
co-registering physiologicwith anatomical data to support non-palpable lesion biopsies, 
radiotheraphy planning and then treatment follow-up. The devices like Breast-specific
gamma imaging which dedicated to small field of view (FOV) and also have emerged 
internationally. For bothe BSGI, and scintimammography, planar orientations (anterior, 
lateral, oblique) are obtained within 5-15 min post-injection.Oblique and lateral images are 
acquired as the patient lies in prone position with pendent breasts while supine positioning 
is supported for oblique, anterior and SPECT tomography acquisitions. In both positions 
image acquisitions are preferred. Prone positioning better seperates breast tissue from high 
pharmaceutical uptake in the myocardium and liver improving visualization of breast activity 
by reducing scattering of photon and improving contrast of image. Additionally , advantages 
of prone position which is enhanced evaluation of the chest wall, better outline of the 
breast contour and improved spatial resolution. While in supine position improves 
visualization of primary lesion and internal mammary. BSGI obtains 210 mini imagesin the 
craniocaudial and mediolateral oblique orientations per breast. For great importance to 
diagnosis and prognosis of patient is breast lesion size and palpability as small non-palpable 
lesions which indicates early disease. Characterising mammographically non-palpable 
benign lesions as metabolically benign could obviate the need for a difficult biopsy
procedure and instead support clinical observation. The high specificity of 
scintimammography , a positive scintigraphic finding would support recommending an 
invasive evaluation. While multiple studies suggested that dedicated combined with breast 
positioning during BSGI provided better detection of sub-centimeter and non palpable 
lesions , this is compared to scintimammography, overall sensitivity and specificity was 82%
and 85% respectively and also showing no improvement over SPECT, which was 86 and 87 
percentages respectively. Indeed, the dedicated camera design of BSGI is highly sensitive for 
detecting local disease, but is limited in the broader clinical setting compared to planar, 
SPECT, and SPECT/CT cameras that can investigate regional, axial and global disease. 
\end{document}
