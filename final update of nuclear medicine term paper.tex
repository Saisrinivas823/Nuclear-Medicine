\documentclass{article}
\usepackage[utf8]{inputenc}

\title{Role of nuclear medicine in breast cancer}
\author{Submitted By Lingamgunta saikumar}
\date{09, march 2022}

\begin{document}

\maketitle
\section{Abstract}
In the world, this breast cancer is the most popular and growing disease especially in 
present situations. By early detection only we can prevent this cancer. And there are so 
many cases that are handled by early detection and decrease the rate of death. And also 
many research works have been done on this breast cancer. In this paper we are discussing 
about detection of breast cancer through nuclear medicine . Routine nuclear medicine is a 
major contribution to a complete range of clinical studies such as early lesion detection and 
stratification; monitoring , guiding, progression in monitoring, recurrence or metastases and 
predicting response to therapy. Particularly recent developed instrumentation such as highresolution device coupled with the diagnostic versatility of conventional cameras have 
reinserted nuclear medicine as a valuable tool in the broader clinical setting. This shows the 
outline of nuclear medicine, concluding that targeted radiopharmaceuticals and versatile 
instrumentation position nuclear medicine as a powerful modality for patients with breast 
cancer.
\section{Introduction}
In the world, this breast cancer is the most popular and growing disease especially in 
present situations. By early detection only we can prevent this cancer. And there are so 
many cases that are handled by early detection and decrease the rate of death. And also 
many research works have been done on this breast cancer. In this paper we are discussing 
about detection of breast cancer through nuclear medicine . Routine nuclear medicine is a 
major contribution to a complete range of clinical studies such as early lesion detection and 
stratification; monitoring , guiding, progression in monitoring, recurrence or metastases and 
predicting response to therapy. Particularly recent developed instrumentation such as highresolution device coupled with the diagnostic versatility of conventional cameras have 
reinserted nuclear medicine as a valuable tool in the broader clinical setting. This shows the 
outline of nuclear medicine, concluding that targeted radiopharmaceuticals and versatile 
instrumentation position nuclear medicine as a powerful modality for patients with breast 
cancer.
While changing our lifestyle factors such as excessive alcohol consumption, obesity and 
physically inactive have also contributed to increased the rate of death from breast cancer. 
Mortality rate which is caused by breast cancer is currently the second leading cause of 
death for women in developed regions. Since 2008, the support of clinical approaches that 
decreases the risk and increase early detection and treatment through worldwide, while 
frequency increases in 20% and rise in mortality by 14%. When incorporating nuclear 
medicine technology in the diagnostic plan, it can readily achievable by multimodality 
approach by using tailored management. In instruments the developments has been taking 
place such as high resolution dedicated breast device and it has coupled with the diagnostic 
versatility of conventional cameras have reinserted nuclear medicine as a valuable tool in 
clinical setting in broader range. This shows general nuclear medicine is a critical role in the 
care of women with breast cancer including detection and stratification, guiding treatment
by using sentinel node imaging (which is used to determine whether cancer has spread 
beyond a primary tumor into your lymphatic system), monitoring cardiotoxicity from 
therapeutic regimens and evaluating local and global progression. In addition with that, it 
describes the role of positron emission tomography or computed tomography is discussed.
\section{Scintimammography}
Scintimammography in nuclear medicine, which utilizes a wide range of instrumentation 
applications . Especially in recent years, conventional planar scintimammography has been 
enhanced by single-photon has been enhanced by single-photon emission computed 
tomography (SPECT) and hybrid SPECT/CT . While hybrid SPECT/CT adds clinical benefit by 
combining physiologic and anatomical data to facilitate non-palpable lesion biopsies, 
radiotherapy planning, and treatment monitoring. Breast-specific gamma imaging systems, 
which are specialised to a limited field of view (FOV), have also become popular 
internationally. Planar orientations (anterior, lateral, oblique) are obtained within 5-15 
minutes after injection for both BSGI and scintimammography. The patient is in a prone 
position with pendent breasts for oblique and lateral pictures, whereas supine stance is 
supported for oblique, anterior, and SPECT tomography acquisitions. Image acquisition is 
preferred in both locations. By reducing photon scattering and boosting imaging contrast, 
prone placement better separates breast tissue from high pharmacological uptake in the 
heart and liver, allowing for improved observation of breast activity.
Additionally, the prone posture has advantages such as higher spatial resolution, improved 
evaluation of the chest wall, and better delineation of the breast contour. Supine position 
allows for better visibility of the main lesion and internal mammary. In the craniocaudial and 
mediolateral oblique orientations, BSGI obtains 210 micro pictures per breast. Breast lesion 
size and palpability are extremely important in the diagnosis and prognosis of patients, as 
small non-palpable lesions suggest early disease. Mammographically non-palpable benign 
lesions that are metabolically benign could be classified as such, avoiding the need for a 
laborious biopsy procedure and instead allowing for clinical surveillance. Because of 
scintimammography's great specificity, a positive scintigraphic finding would support a 
recommendation for an invasive assessment.
When compared to scintimammography, which had overall sensitivity and specificity of 82 
percent and 85 percent, respectively, and showed no improvement over SPECT, which had 
86 and 87 percent respectively, multiple studies suggested that dedicated combined with 
breast positioning during BSGI provided better detection of sub-centimeter and nonpalpable lesions. BSGI's unique camera design is highly sensitive for detecting local disease, 
but it has limitations in the broader clinical situation when compared to planar, SPECT, and 
SPECT/CT cameras, which can study regional, axial, and global disease.
\section{Sentinel Lymph Node Scintigraphy}
Axial nodal bed status is the most important prognostic factor for newly diagnosed patients 
with invasive breast cancer, and it's also important for deciding on treatment options. 
Because imaging methods for axillary staging are rarely sensitive or specific, surgical 
exploration of nodal involvement is required. Sentinel lymph node status through biopsy 
(SLNB) was developed as a less intrusive alternative to established staging approaches that 
often increase morbidity, such as axial lymph node dissection (ALND). A histopathologically 
negative sentinel node means that the ipsilateral nodal bed is clear of metastatic illness 
because it is the first relay receiving lymphatic outflow straight from the tumour. The cost36 
and comorbidities associated with preserving the healthy nodal bed are reduced. The cost36 
and common comorbidities associated with ALND are both reduced when the healthy nodal 
bed is spared. Negative sentinel node status is related with a 0–2% axillary recurrence rate.
The use of radiolabelled colloids in nuclear medicine planar (dynamic or static) and/or 
SPECT/CT sentinel node imaging provides surgeons with a visual map to enable correct 
localization of sentinel nodes and unusual drainage patterns. The identification of the 
sentinel node is critical to the success of SLNB, and preoperative sentinel node imaging is 
well adapted for this task, with a detection rate of 94 percent to 100 percent. Sentinel node 
imaging with SPECT/CT of all tumours may give a more reliable technique to locate and 
biopsy sentinel nodes for staging in patients with multicentric and multifocal illness whose 
lymphatic drainage patterns may differ.
Particle size and dose concentration can alter lymphatic transit of radiocolloid particles 
following injection. Large colloid particles' lymphatic movement is often hampered by their 
size, resulting in either delayed or non-visualization of axillary sentinel nodes. Small 
particles, on the other hand, migrate quickly and may not be trapped in the sentinel node.
So that lymphatic physiology is adequately depicted, solvent volumes should be adjusted to 
injection technique. Smaller quantities are better for peritumoural injections, for example, 
because higher pressure can promote leakage into the extravascular space, which can then 
migrate to nearby lymphatic channels.
Preoperative sentinel node mapping is consistently provided by planar acquisitions (Fig. 1), 
and the addition of multimodality SPECT/CT further strengthens this role. Oriented 
SPECT/CT slices allow surgeons to precisely anatomically localise nodal uptake, decreasing 
surgical time and enhancing biopsy accuracy (Fig. 2). Evidence also suggests that SPECT/CT 
can detect difficult-to-interpret drainage patterns and sentinel nodes (such as parasternal 
nodes) that aren't seen on planar imaging, limiting blind exploratory surgery. For precise 
visualisation of lymphatic drainage patterns and labelling sentinel nodes, patient positioning 
and acquisition settings are critical. Patient photos, patient markers, and an auditory gamma 
probe all work together to help surgeons identify the radioactive sentinel node or nodes for 
excision during surgery.
\section{Tc-99m Multigated Radionuclide Angiography}
The use of combination chemotherapy has been well recognised in cancer research to cause 
cardiac problems, necessitating routine monitoring of left ventricular ejection fraction. 
Anthracyclines and the monoclonal antibody trastuzumab are two anti-cancer medications 
with well-known cardiotoxicity among the constantly expanding array of anti-cancer 
therapies used in breast cancer treatment. Anthracycline treatment causes cumulative 
dose-dependent cardiotoxicity in myocytes, which can lead to congestive heart failure and 
even death. When used in combination with anthracyclines, trastuzumab can cause severe 
cardiotoxicity. Furthermore, pericardial irritants such as radiation therapy and 
chemotherapy medications enhance the likelihood of pericardial effusion problems, making 
early diagnosis crucial to reducing the primary or contributing related mortality found in 86 
percent of symptomatic cancer patients.
Coronary artery disease, valve disease, constructive pericarditis, and myocardial 
dysfunction, including congestive heart failure, are all examples of radiation-induced heart 
disease that can severely affect a patient's quality of life.As a result, determining the 
amount of a patient's risk of a cardiac episode is critical. Obtaining ejection fraction 
measures prior to and after chemotherapy treatment in breast cancer patients is the usual 
way for detecting cardiotoxicity and pericardial effusions. Serial examination allows 
clinicians to track a patient's cardiac response to treatment, lowering the likelihood of 
comorbidities caused by chemotherapy.
Figure 1. Sentinel node localisation study of the right breast showing four periareolar 
radiopharmaceutical injections oriented on anterior, right anterior oblique (RAO) and right lateral 
planar images. An intense focal uptake lateral to the injection sites is identified as well as faint 
uptake in three axillary lymphatic chain nodes. Image courtesy of Regional Imaging, a member of IMED Network Radiology, Wagga Wagga, NSW.

Figure 2. Fused SPECT/CT images provide surgeons with a more accurate anatomical visualisation of 
the sentinel node with regard to the breast and axilla by depicting the precise location in the axillary 
area adjacent to the first and second rib saving valuable intraoperative time and increasing surgical 
confidence. SPECT, single-photon emission computed tomography. Image courtesy of Regional 
Imaging, a member of I-MED Network Radiology, Wagga Wagga, NSW
In patients undergoing chemotherapy, multigated radionuclide angiography (RNA) or 
equilibrium radionuclide angiocardiography (ERNA) is considered the gold standard for 
measuring heart function with good repeatability and low inter-observer variability. RNA is a 
noninvasive technology that measures regional and global wall motion, ventricular systolic 
and diastolic function (both right and left ventricular ejection fractions), and ventricular 
volumes using Tc-99m pertechnetate erythrocyte labelling. In vivo, in vitro, or modified in 
vivo procedures are used to label red blood cells with Tc-99m pertechnetate, as 
documented in the literature. The acceptable level of picture quality, patient throughput 
needs, and technical staff knowledge will all influence the choice of a blood pool agent in 
any given clinical circumstance.
Early diagnosis of cardiotoxic abnormalities on serial imaging allows for prompt 
intervention, reducing the risk of related patient morbidity or mortality from trastuzumab 
and anthracycline therapy. The use of RNA to assess global LV systolic function and diastolic 
performance indexes has been shown to be effective for early detection of functional 
changes after chemotherapy when compared to baseline, highlighting the importance of 
serial imaging at all stages of treatment to assess patient prognosis. A decrease in peak fill 
rate as measured by multigated RNA implies poor diastolic function, which occurs before 
systolic function declines in anthracycline-induced cardiotoxicity and, more critically, is an 
early indicator of compromised heart function. Pericardial effusions in small or moderate 
amounts may not impair LVEF, although they can cause comorbidities such as difficulty 
breathing and caused chest discomfort.
\section{Positron Emission Tomography/ Computed Tomography}
Oncologic investigations accounted for 94 percent of the projected 1.5 million PET/CT 
procedures performed in the United States in 2011. F-18 fluoro-2-deoxyglucose (F-18 FDG), 
a tumor-avid glucose analogue, has acquired general acceptance as a marker of cellular 
metabolism, providing molecular insight into cancer physiology. PET/ability CT's to precisely 
stage WB metastatic disease as well as quantify and evaluate therapy response has had a 
huge impact on how breast cancer patients are treated.
Because the method may not sufficiently resolve small primary breast tumours or axillary 
nodal areas, a PET/ CT with high sensitivity and specificity is optimal for detecting distant 
metastases. Dedicated positron emission mammography (PEM) devices were created to 
increase detection of small primary lesions by maximising system spatial resolution, and are 
currently being investigated to address this problem. PET/CT is comparable to and may 
surpass whole body bone scan (WBBS) in the diagnosis of breast cancer metastases to bone, 
with similar sensitivity but higher specificity due to superior metabolic and morphologic 
characterization of osseous lesions.
In many treatment contexts, F-18 FDG can also give an earlier and more objective prediction 
and assessment of response. Serial standard uptake value (SUV) measurements are 
produced directly from tumour metabolic avidity using a standardised technique, and so 
provide a semi-quantitative depiction of cancer physiology across time. A greater lesion 
SUVmax, including values acquired from osseous lesions, also has a favourable connection 
with prognostically poor aggressive illness. PET/CT has also been used to investigate new 
positron-emitting radiopharmaceuticals that target oestrogen, progesterone, oestrogen 
growth factor, and somatostatin receptors found on breast tumours to improve specificity. 
Receptor-mediated molecular imaging aids tumour characterization, customised 
therapeutic research, disease burden extent, and disease monitoring through serial imaging.
\section{Conclusion}
Nuclear medicine is a vital component of illness diagnosis, therapy, and prognosis because it 
fills a physiologic gap in medical imaging for cancer patients, delivering information that is 
representational of function.
General nuclear medicine, in particular for breast cancer care, has diagnostic value at every 
stage of the disease and is considered the first-line modality in a variety of clinical settings. 
The adaptability of gamma cameras with SPECT and/or SPECT/CT is an appropriate tool for 
patients in a broader clinical setting, providing a very detailed view into the current 
physiologic state of bodily structures and their functions. PET/CT has established itself as a 
powerful oncologic tool, and with the introduction of tailored radiopharmaceuticals, it is 
expected to see a significant increase in its use in the treatment of breast cancer.
Future research should look at multi-acquisition (planar and SPECT) and multi-modality 
(SPECT/CT) imaging in the full scope of physiologic breast cancer imaging to better 
understand the added utility of instrument hybridisation methods and processes in nuclear 
medicine imaging.
\section{References}
1. International Agency for Research on Cancer. Breast cancer estimated incidence, mortality 
and prevalence worldwide in 2012. World Health Organization [updated 2012]. Available 
from: http://www.globocan.iarc.fr/Pages/fact_sheets_cancer.aspx (accessed 8 January 
2015).
2. Shah R, Rosso K, Nathanson SD. Pathogenesis, prevention, diagnosis and treatment of 
breast cancer. World J Clin Oncol 2014; 5: 283–98.
3. Brem R, Rechtman L. Nuclear medicine imaging of the breast: a novel, physiological 
approach to breast cancer detection and diagnosis. Radiol Clin North Am 2010; 48: 1055–74. 
4. Munnink TH, Nagengast WB, Brouwers AH, et al. Molecular imaging of breast cancer. Breast 
2009; 18: S66–73.
5. Ferrara A. Nuclear imaging in breast cancer. Radiol Technol 2010; 81: 233–46. 
Term paper
6. Tiling R, Kebler M, Untch M, Sommer H, Linke R, Hahn K. Initial evaluation of breast cancer 
using Tc-99m sestamibi scintimammography. Eur J Radiol 2005; 53: 206–12. 
7. de Cesare A, Giuseppe DV, Stefano G, et al. Single photon emission computed tomography 
(SPECT) with Technetium-99m sestamibi in the diagnosis of small breast cancer and axillary 
lymph node involvement. World J Surg 2011; 35: 2668–72.
8. Mettler F, Guiberteau M. Essentials of Nuclear Medicine Imaging, 6th edn. Mosby Elsevier, 
St. Louis, MO, 2012.
9. Specht JM, Mankoff DA. Advances in molecular imaging for breast cancer detection and 
characterization. Breast Cancer Res 2012; 14: 206–17. 
10. Hendrick RE. Radiation doses and cancer risks from breast imaging studies. Radiology 2010; 
257: 246–53.
11. Taillefer R. Clinical applications of 99mTc-sestamibi scintimammography. Semin Nucl Med 
2005; 35: 100–15. 
12. Jacobsson H. Single-photon-emission computed tomography (SPECT) with 99mTechnetium 
sestamibi in the diagnosis of small breast cancer and axillary node involvement. World J Surg 
2011; 35: 2673–4. 
13. Kong FL, Kim E, Yang D. Targeted nuclear imaging of breast cancer: status of radiotracer 
development and clinical applications. Cancer Biother Radiopharm 2012; 27: 105–12. 
14. Lee J, Rosen E, Mankoff D. The role of radiotracer imaging in the diagnosis and management 
of patients with breast cancer: part 1 – overview, detection and staging. J Nucl Med 2009; 
50: 569–81. 
15. Nguyen BD, Roarke MC, Karstaedt PJ, Ingui CJ, Ram PC. Practical applications of nuclear 
medicine in imaging breast cancer. Curr Probl Diagn Radiol 2009; 38: 68–83. 
16. Kim SJ, Kim IJ, Bae YT, Kim YK, Kim DS. Comparison of quantitative and visual analysis of Tc99m MIBI scintimammography for detection of primary breast cancer. Eur J Radiol 2005; 53: 
192–8. 
17. Prekeges J. Breast imaging devices for nuclear medicine. J Nucl Med Technol 2012; 40: 71–8.
18. Waxman A. The role of 99mTc methoxyisobutylisonitrile in imaging breast cancer. Semin 
Nucl Med 1997; 27: 40–54. 
19. Silvera S, Rohan T. Benign proliferative epithelial disorders of the breast: a review of the 
epidemiologic evidence. Breast Cancer Res Treat 2008; 110: 397–409. 
20. Sergieva S, Alexandrova E, Baitchev G, Parvanova V. SPECT-CT in breast cancer. Arch Oncol 
2012; 20: 127–31.
21. Whitman G, Strom E. Workup and staging of locally advanced breast cancer. Semin Radiat 
Oncol 2009; 19: 211– 21. 
22. Glendenning J, Cook G. Imaging breast cancer bone metastases: current status and future 
directions. Semin Nucl Med 2013; 317–23. 
23. Iqbal B, Currie G, Wheat J, Raza H, Ahmed B, Kiat H. Incremental value of SPECT/CT in 
characterizing solitary spine lesions. J Nucl Med Technol 2011; 39: 201–7.
24. Fink C, Hasan B, Deleu S, Pallis A, Baas P, O’Brien M. High prevalence of osteoblastic bone 
reaction in computed tomography scans of an European Organisation for Research and 
Treatment of Cancer prospective randomised phase II trial in extensive stage small cell lung 
cancer. Eur J Cancer 2012; 48: 3157–60. 
25. IMVinfo.com. [homepage on the Internet]. IMV 2012 PET imaging market summary report 
[updated 2012 Aug 7]. Available from: http://www.imvinfo. 
Term paper
com/user/documents/content_documents/def_dis/2012_ 
08_07_13_51_48_43_IMV_PET2012_report_ datasheet.pdf(accessed 7 January 2015). 
26. Gallamini A, Zwarthoed C, Borra A. Positron emission tomography (PET) in oncology. 
Cancers 2014; 6: 1821–89. 
27. Bourgeois A, Warren L, Chang T, Embry S, Hudson K, Bradley Y. Role of positron emission 
tomography/ computed tomography in breast cancer. Radiol Clin North Am 2013; 51: 781–
98. 
28. Hong S, Li J, Wang S. 18FDG PET-CT for diagnosis of distant metastases in breast cancer 
patients. A metaanalysis. Surg Oncol 2013; 22: 139–43. 
29. Bensch F, van Kruchten M, Lamberts L, Schroder C, Hospers G, Brouwers A, van Vugt M, de 
Vries E. Molecular imaging for monitoring treatment response in breast cancer patients. Eur 
J Pharmacol 2013; 171: 2–11. Nuclear Medicine and Breast Cancer L. R. Greene et al. 64 ª 
2015 The Authors. Journal of Medical Radiation Sciences published by Wiley Publishing Asia 
Pty Ltd on behalf of Australian Institute of Radiography and New Zealand Institute of 
Medical Radiation Technology 
30. Kalles V, Zografos GC, Provatopoulou X, Koulocheri D, Gounaris A. The current status of 
positron emission mammography in breast cancer diagnosis. Breast Cancer 2013; 20: 123–
30.
\end{document
