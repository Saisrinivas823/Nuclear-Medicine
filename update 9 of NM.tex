\documentclass{article}
\usepackage[utf8]{inputenc}

\title{Role of nuclear medicine in breast cancer}
\author{Submitted By Lingamgunta saikumar}
\date{18, march 2022}

\begin{document}

\maketitle

\section{Sentinel Lymph Node Scintigraphy}

Axial nodal bed status is the most important prognostic marker for newly diagnosed patients with 
invasive breast cancer, and it is also important for deciding suitable treatment. Because imaging 
methods for axillary staging are rarely sensitive or specific, nodal involvement must be investigated 
surgically. Sentinel lymph node status by biopsy (SLNB) was introduced as a less invasive alternative 
to classic staging approaches such axial lymph node dissection (ALND), which can predict nodal 
involvement with good accuracy. A histo-pathologically negative sentinel node indicates that the 
ipsilateral nodal bed is clear of metastatic illness because it is the first relay receiving lymphatic 
outflow straight from the tumour.
The use of radiolabelled colloids in nuclear medicine planar (dynamic or static) and/or SPECT/CT 
sentinel node imaging provides surgeons with a visual map to enable correct localization of sentinel 
nodes and unusual drainage patterns. The identification of the sentinel node is critical to the success 
of SLNB, and preoperative sentinel node imaging is well adapted for this task, with a detection rate 
of 94 percent to 100 percent. Sentinel node imaging with SPECT/CT of all tumours may give a more 
reliable technique to locate and biopsy sentinel nodes for staging in patients with multicentric and 
multifocal illness whose lymphatic drainage patterns may differ.
Periareolar, peritumoural, subdermal, subareolar, and other radiocolloid injection procedures are 
used. Intradermal, intratumoural, and subtumoural treatments are available. The periareolar and 
subareolar injection sites are the two most prevalent injection sites.
(superficial) via the lymphatic-rich subareolar space plexus, which allows for fast drainage 
visualisation. peritumoural channels with high target count rates (deep), which is capable of tracing 
additional drainage pattern. Specifically, the intra-mammary chain, which is found in 20–30% of 
women. There is evidence that the injection location is because time-to-visualisation rates appear to 
be influenced rather than determined, it is thought to be rather adaptable. Rates of false negatives. 
Variables are also mentioned in the literature. The ability to forecast is influenced by lymphatic 
drainage patterns. which lymphatic arteries would be responsible for draining the tumour and thus 
should be taken into account when selecting an injection technique.
\end{document}
