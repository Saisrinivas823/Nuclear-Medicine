\documentclass{article}
\usepackage[utf8]{inputenc}

\title{Role of nuclear medicine in breast cancer}
\author{Submitted By Lingamgunta saikumar}
\date{4, feb 2022}

\begin{document}

\maketitle

\section{Scintimammography}
It is the best used in clinical applications where the mainstay modalities of anatomic mammography and ultrasound are limited and mainly it provides the non-invasive in vivo characterization of infectious diseases from benign processes. Targeted uptake improves diagnostic integrity and decreases the false-positives rate, which is commonly seen in anatomical imaging which in turn reduces unnecessary invasive procedures and biopsies. Technetium-99m sestamibi, it is a radiopharmaceutical of choice for studying single-photon especially in breast imaging due in part to an overall tumour to background ratio of 6:1. The recommended administered dose range is 740-1100 MBq (20-30 mCi), resulting in an absorbed dose between 40.0 and 55.5 mGy to the large intestine (critical organ), 20 mGy to the kidney, bladder wall, and gallbladder wall each, and 2 mGy to the kidney, bladder wall and gallbladder wall each and 2 mGy to the breast (lowest dose).An effective dose (a summed entire body exposure estimate that is weighted based on tissue radiosensitivity) was estimated to be 5.9–9.4 mSv compared to 0.44 mSv for digital mammography. As much as 90% of Tc-99m sestamibi dose is localised to the dense mitochondria characteristic of malignant cells, with uptake dependent on regional blood flow, tumour angiogenesis, increased metabolism and driven by plasma membrane potentials and mitochondrial membrane potentials. Tumour efflux of Tc-99m sestamibi has been correlated with cellular expression of P-glycoprotein, a protective transmembrane protein pump found in cells that over express the multi-drug resistant gene. These findings are clinically relevant as Tc-99m sestamibi can provide an in vivo prediction of anticancer drug efficacy. Moreover, faster rates of Tc-99m sestamibi efflux have been associated with higher levels of P-glycoprotein expression. Uptake of Tc-99m sestamibi may be non-specific, for example, in sites of prior surgical intervention or inflammation, and as such increased perfusion to these sites may result in false-positives. False-positive uptake has also been associated with benign diseases such as hyper-proliferative breast disease and atypical hyperplasia. A higher predilection towards malignant transformation has been demonstrated with both pathologies, speculating that a false-positive finding in these patients may reflect premalignant potential. 
\end{document}
