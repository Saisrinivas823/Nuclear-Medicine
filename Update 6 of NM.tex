\documentclass{article}
\usepackage[utf8]{inputenc}

\title{Role of nuclear medicine in breast cancer}
\author{Submitted By Lingamgunta saikumar}
\date{25, Feb 2022}

\begin{document}

\maketitle

\section{Tc-99m Multigated Radionuclide Angiography}
It has been well documented throughout cancer research that the use of multidrug 
chemotherapy causes heart failure and requires regular monitoring of the left ventricular 
ejection fraction. Among the fast-growing anti-cancer drugs used to treat breast cancer, 
anthracyclines and monoclonal antibody trastuzumab, agents have a well-known 
cardiotoxicity. Anthracycline chemotherapy causes an increase in dose-dependent 
cardiotoxicity with direct and irreversible cell damage to the myocytes, which can lead to 
heart failure and cardiac death. Trastuzumab can cause high levels of cardiotoxicity 
especially when combined with anthracyclines. In addition, radiation therapy and 
chemotherapy irritate pericardials and thus increase the risk of pericardial effusion 
problems, making early diagnosis critical for the reduction of primary or underlying related 
deaths seen in ~ 86% of patients with symptomatic cancer. Cardiovascular disease such as 
coronary artery disease, valvular disease, constructive pericarditis and myocardial 
dysfunction involving severe heart failure can significantly reduce a patient's quality of life. 
Therefore, it is important to effectively determine the patient's level of risk for a cardiac 
event. The most common way to detect cardiac toxicity and pericardial infarction is to 
obtain fractional extracts made before and after chemotherapy in patients with breast 
cancer. Serial testing allows physicians to monitor a patient's heart response to treatment 
that reduces the risk of chemotherapy-induced comorbidities.
Multigated radionuclide angiography (RNA) or equilibrium radionuclide angiocardiography 
(ERNA) is considered a gold standard for measuring cardiac function with high frequency 
and low spectrum variability in chemotherapy patients. RNA is a non-invasive method that 
uses the Tc-99m pertechnetate erythrocyte label to test regional and global wall 
movements, ventricular systolic and diastolic function (both right and left ventricular 
ejection components), and ventricular volumes. Labeling of red blood cells with Tc-99m 
pertechnetate was performed using in vivo, in vitro, or in vivo modification as described in 
the literature. In any clinical setting, the selection of a blood pool agent will depend on the 
acceptable level of image quality, patient discharge requirements, and the professional level 
of the technical staff. Patient acceptance may also influence product selection. For example, 
some patients may reject in vitro so-called red blood cells based on religious beliefs about 
transfusions.
\end{document}
