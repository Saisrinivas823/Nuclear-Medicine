\documentclass{article}
\usepackage[utf8]{inputenc}

\title{Role of nuclear medicine in breast cancer}
\author{Submitted By Lingamgunta saikumar}
\date{11, march 2022}

\begin{document}

\maketitle

\section{Whole Body Bone Scan in Metastatic Disease }
Bone metastases are present in 80% of four women metastatic disease that 
highlights breast speculation cancer cells to metabolize bone marrow. Once a 
bone metastases found, average survival is between 2.1 and 6 years. 
Additionally, patients may experience a deterioration of their quality of life 
from related complications of metastatic disease. As a result, the first stage 
and the standard follow-uptest recovery is important in improving quality of 
life as well survival. WB scan is a less expensive testing and not importing 
that accurate measurement of disease burden is recommended and followup tests with symptoms or high risk patients.
WB bone scanning using plan, SPECT, and / or SPECT / CT is the first powerful 
tool for stage and response therapy. WB bone scintigraphy uses bone that 
seeks out radiopharmaceuticals found in the genital area increased perfusion 
and bone changes feature bone metastases making it a very sensitive tool 
finding mixed osteoblastic and lytic / sclerotic lesions. Adding data for 
different components from SPECT has shown to detect more than planar 
lesions as well improve clarity especially on the axial skeleton there wound 
healing is difficult to discriminate against or otherwise degenerative 
absorption. In addition, integration hybrid SPECT / CT data can better 
diagnose abnormalities tracer detection in benign skeletal pathology as 
degenerative changes or cysts. Registered data and can confirm lytic ulcers, 
which if repaired the process that is done will appear cold in the edited 
images and thus it has been difficult to see. The study of Iqbal et al. received 
an improvement in 6.1% to 78.8% on an assessment of the accuracy of 
vertebral separation metastases when adding SPECT / CT data to the planar.
the predictable number of jointly registered SPECT / CT can be identified 
results from a recent survey of 33.8% and increased by 2.1% of breast cancer 
patients there compared to planar and SPECT bone scan. Systemic testing 
(chemotherapy or hormone therapy) or a local (radiotherapy) treatment 
response is local where nuclear drugs can thrive. WB bone scanning provides 
pre-existing and other information as it stands indicates the body's response 
to treatmentintervention also generally provides a better predictable value
there is an anatomical response. In some cases, bone Scintigraphy may 
highlightthe osteoblastic bone reaction commonly referred to as the bone 
flare phenomenon which is characterized by an increase in radiotracer 
detection as a result of extended treatment of osteoblastic activity showing 
cooling bone. Flare appears in WB bone scan as an exaggerated appearance 
of the disease or representing a healing or previously invisible bone ulcers. In
both cases, the appearance of the flare response to bone scintigraphy may 
provide important information prognostic information indicating healing or 
the progression of the disease.
\end{document}
