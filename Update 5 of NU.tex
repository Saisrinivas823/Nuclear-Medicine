\documentclass{article}
\usepackage[utf8]{inputenc}

\title{Role of nuclear medicine in breast cancer}
\author{Submitted By Lingamgunta saikumar}
\date{18, Feb 2022}

\begin{document}

\maketitle

\section{Positron emission Tomography/Computed tomography}
Especially in the United states, the studies of oncologic accounted for 
approximately 94% of the estimated 1.5 million positron emission 
tomography/ computed tomography procedures performed in 2011. F-18 
fluoro-2-deoxyglucose(F-18FDG), which is known as tumour-avid glucose 
analog has gained widespread acceptance as a marker of cellular metabolism 
providing insight into cancer physiology at the molecular level. The ability of 
Positron emission tomography/ computed tomography to accurately stage WB 
metastatic disease as well as quantify and evaluate the responses from 
therapeutic conditions, it has significantly embedded and tailored the care 
breast cancer patients receive. High sensitivity and specificity best appropriate 
positron emission tomography/ computed tomography for detecting distant 
metastases as the system may not adequately resolve small primary beast 
lesions or axillary nodal regions. At present under investigation to address this 
limitation, dedicated positron emission mammography devices were actually 
designed to improve the detection of small primary lesions by maximising 
system spatial resolution. PET / CT is compared and in some cases may surpass 
whole body body scan (WBBS) in detecting breast cancer bone metastases, 
showing similar sensitivity but greater specificity as a result of improved 
metabolic manifestations and morphologic osseous lesions. The F-18 FDG may 
also provide preliminary and purposeful prediction and response testing in 
many treatment settings. Using a fixed protocol, serial standard uptake value 
(SUV) measurements are obtained directly from tumor metabolic avidity and 
thus provide a limited representation of cancer science over time. There is also 
a positive correlation between the severity of the infection and the SUVmax of 
the upper extremity, which includes the values obtained for osseous lesions. 
Novel positron-emitting radiopharmaceuticals designed to improve specificity 
by targeting estrogen, progesterone, estrogen growth factor and somatostatin 
receptors detected in breast tumors were also tested by PET / CT. Plant 
identification, concomitant treatment investigations, disease load level and 
disease monitoring by serial imaging are facilitated by receptor-mediated 
molecular imaging.
\end{document}
